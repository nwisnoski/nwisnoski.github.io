%%%%%%%%%%%%%%%%%%%%%%%%%%%%%%%%%%%%%%%%%
% Medium Length Professional CV
% LaTeX Template
% Version 2.0 (8/5/13)
%
% This template has been downloaded from:
% http://www.LaTeXTemplates.com
%
% Original author:
% Trey Hunner (http://www.treyhunner.com/)
%
% Important note:
% This template requires the resume.cls file to be in the same directory as the
% .tex file. The resume.cls file provides the resume style used for structuring the
% document.
%
%%%%%%%%%%%%%%%%%%%%%%%%%%%%%%%%%%%%%%%%%

%-------------------------------------------------------------------------------
%	PACKAGES AND OTHER DOCUMENT CONFIGURATIONS
%-------------------------------------------------------------------------------

\documentclass{resume} % Use the custom resume.cls style

\usepackage[left=.75in,top=1in,right=.75in,bottom=1in]{geometry} % Document margins
\usepackage{fancyhdr}
\usepackage{datetime}
\usepackage{eurosym}
\usepackage{hyperref}
\usepackage{hanging}
\usepackage{csquotes}
\newdateformat{mydate}{\THEMONTH/\THEYEAR}
\fancyhf{} % sets both header and footer to nothing
\renewcommand{\headrulewidth}{0pt}
\rfoot{\mydate\today}
\cfoot{\thepage}



\name{\sc Nathan I. Wisnoski} % Your name
\address{Department of Biology, Indiana University \\ 1001 E. Third Street, Bloomington, IN 47405} % Your address
\address{wisnoski@indiana.edu \\ \url{nwisnoski.github.io}} % Your phone number and email

\begin{document}

%-------------------------------------------------------------------------------
%	EDUCATION SECTION
%-------------------------------------------------------------------------------

\begin{rSection}{Education}

{\bf Indiana University, Bloomington} \hfill {2014 -- Present} \\
Ph.D in Biology -- Evolution, Ecology, and Behavior \\
Minor in Environmental Sciences, School of Public and Environmental Affairs

{\bf The University of Texas at Austin} \hfill {2009 -- 2013} \\
B.S. in Biology -- Ecology, Evolution, and Behavior \\
Minor in Business, McCombs School of Business \\

\end{rSection}

%-------------------------------------------------------------------------------
%	RESEARCH EXPERIENCE SECTION
%-------------------------------------------------------------------------------

\begin{rSection}{Research Experience}

% At Indiana University

\begin{rSubsection}{Ph.D. Candidate}{August 2014 -- Present}{Indiana University}{Bloomington, IN}
\item Lennon Lab: Microbial Ecology and Evolution
\end{rSubsection}

\begin{rSubsection}{Field/Lab Technician}{Spring 2014}{University of Texas}{Austin, TX}
\item Hawkes Lab: Microbial Ecology and Biogeochemistry
\end{rSubsection}

\begin{rSubsection}{Undergraduate Researcher}{June 2012 -- February 2014}{University of Texas}{Austin, TX}
\item Leibold Lab: Community and Evolutionary Ecology
\end{rSubsection}

\end{rSection}

\bigskip

%-------------------------------------------------------------------------------
%   Grants
%-------------------------------------------------------------------------------
\begin{rSection}{Grants}

\begin{Grant}{NSF LTER Network Communications Office (NCEAS)}{A synthesis to identify how metacommunity dynamics mediate community responses to disturbance across the ecosystems represented in the LTER network}{PI: E.R. Sokol, co-PIs: C.M. Swan, {\bf N.I. Wisnoski}}{\$76,000}{2016--2018}
\end{Grant}

\begin{Grant}{IU Sustainability Research Development Grant}{How does hydrological flow rate alter bioremediation of pesticides at the surface- groundwater interface?}{PI: {\bf N.I. Wisnoski}}{\$5400}{2015}
\end{Grant}

\end{rSection}

\bigskip

%-------------------------------------------------------------------------------
%   FUNDING/AWARDS/Fellowships
%-------------------------------------------------------------------------------

\begin{rSection}{Fellowships and Awards}

\begin{Award}{Travel Award, Association for the Sciences of Limnology and Oceanography}{\$606}{2018}
\end{Award}

\begin{Award}{Travel Award, ESA Microbial Ecology Section}{\$600}{2017}
\end{Award}

\begin{Award}{George W. Brackenridge Fellowship, IU Biology}{\$2000}{2016}
\end{Award}

\begin{Award}{Travel Award, International Society for Microbial Ecology}{\euro{}300}{2016}
\end{Award}

\begin{Award}{Travel Award, Honorable Mention, ESA Microbial Ecology Section}{\$150}{2016}
\end{Award}

\begin{Award}{Departmental Research Recruitment Fellowship, IU Biology}{ }{2014}
\end{Award}

\end{rSection}


\bigskip
\newpage
%-------------------------------------------------------------------------------
%   Manuscripts
%-------------------------------------------------------------------------------

\begin{rhangSection}{Manuscripts}

\begin{Prep}{{\bf Wisnoski, N.I.}, M.A. Leibold, and J.T. Lennon}{In review}{Dormancy in metacommunities}{\url{https://doi.org/10.17605/OSF.IO/UJMZC}}
\end{Prep}

\begin{Prep}{Voelker, N.M., P.L. Zarnetske, {\bf N.I. Wisnoski}, J.D. Tonkin, C.M. Swan, S. Record, L. Marazzi, N. Lany, T. Lamy, A. Compagnoni, M.C.N. Castorani, R. Andrade, and E.R. Sokol}{In review}{Novel insights to be gained from applying metacommunity theory to long-term biodiversity data}{}
\end{Prep}

\smallskip
{ In preparation:}

\begin{Prep}{{\bf Wisnoski, N.I.} and J.T. Lennon}{In preparation}{Habitat-specific community assembly in a dendritic metacommunity}{}
\end{Prep}

\begin{Prep}{{\bf Wisnoski, N.I.}, M.E. Muscarella, M.L. Larsen, A.L. Peralta, and J.T. Lennon}{In preparation}{Dormancy and dispersal structure bacterial communities across ecosystem boundaries}{}
\end{Prep}

\begin{Prep}{{\bf Wisnoski N.I.} and J.T. Lennon}{In preparation}{The contribution of \enquote{seed banks} to bacterial community dynamics}{}
\end{Prep}

\begin{Prep}{Lamy, T., {\bf N.I. Wisnoski}, R. Andrade, M.C.N. Castorani, A. Compagnoni, N. Lany, L. Marazzi, S. Record, C.M. Swan, J.D. Tonkin, N. Voelker, S. Wang, P.L. Zarnetske, and E.R. Sokol}{In preparation}{The dual dimensions of metacommunity stability}{}
\end{Prep}

\begin{Prep}{Mueller, E.A., {\bf N.I. Wisnoski}, A.L. Peralta, J.T. Lennon}{In preparation}{Microbial rescue effects: how microbiomes can save hosts from extinction}{}
\end{Prep}

\begin{Prep}{Ward, A.S., S.M. Wondzell, N.M. Schmadel, S. Herzog, J.P. Zarnetske, and 33 others}{In preparation}{Spatial and temporal variation in river corridor exchange across a 5th order mountain stream network}{}
\end{Prep}

\end{rhangSection}

\bigskip

%-------------------------------------------------------------------------------
%   Book Reviews
%-------------------------------------------------------------------------------

\begin{rhangSection}{Book Reviews}

\begin{Publication}{{\bf Wisnoski, N.I.} and J.T. Lennon}{2016}{\enquote{Principles of Microbial Diversity} by James W. Brown}{The Quarterly Review of Biology. \url{https://doi.org/10.1086/685351}}
\end{Publication}

\end{rhangSection}

\bigskip

%-------------------------------------------------------------------------------
%   POSTERS / PRESENTATIONS
%-------------------------------------------------------------------------------
\begin{rhangSection}{Invited Talks}

  \begin{Presentation}{{\bf Wisnoski, N.I.}, M.A. Leibold, J.T. Lennon}{Submitted, 2019}{Dormancy in metacommunities: when can temporal dispersal maintain diversity in variable landscapes?}{Society for Freshwater Science Annual Meeting}{Salt Lake City, UT}
  \end{Presentation}
  
  \begin{Presentation}{{\bf Wisnoski, N.I.} and J.T. Lennon}{2018}{Contribution of \enquote{seed banks} to bacterioplankton community dynamics}{Society for Freshwater Science Annual Meeting}{Detroit, MI}
  \end{Presentation}

\end{rhangSection}

\bigskip

\begin{rhangSection}{Contributed Talks and Posters}
  
    \begin{Presentation}{{\bf Wisnoski, N.I.}, E.R. Sokol, R. Andrade, M.C.N. Castorani, C.P. Catano, A. Compagnoni, T. Lamy, N.K. Lany, L. Marazzi, S. Record, A.C. Smith, C.M. Swan, J.D. Tonkin, N.M. Voelker, P.L. Zarnetske}{Submitted, 2019}{Patterns and drivers of stability in long-term metacommunity data}{Ecological Society of America Annual Meeting}{Louisville, KY}
  \end{Presentation}
  
  \begin{Presentation}{Ward, A.S., S. Herzog, S.M. Wondzell, N. Schmadel, P. Blaen, J. Drummond, D.M. Hannah, C.J. Harman, J. Knapp, S. Krause, M.J. Kurz, A. Li, E. Marti, M. Miller, A. Milner, K. Neil, S. Plont, K. Roche, A.I. Packman, {\bf N. Wisnoski}, J. Zarnetske}{2018}{How do hydrologic forcing and geologic setting control river corridor exchange in a 5th order mountain stream network?}{Geological Society of America Annual Meeting}{Indianapolis, IN}
  \end{Presentation}
  
  \begin{Presentation}{Sokol, E.R., {\bf N.I. Wisnoski}, C.M. Swan}{2018}{Using long-term data to understand when metacommunities respond to disturbance}{Ecological Society of America Annual Meeting}{New Orleans, LA}
  \end{Presentation}

  \begin{Presentation}{{\bf Wisnoski, N.I.}, M.E. Muscarella, and J.T. Lennon}{2018}{Dispersal and dormancy across ecosystem boundaries}{Association for the Sciences of Limnology and Oceanography}{Victoria, BC, Canada}
  \end{Presentation}

  \begin{Presentation}{{\bf Wisnoski, N.I.} and J.T. Lennon}{2017}{Microbial community assembly in dendritic metacommunities}{Ecological Society of America Annual Meeting}{Portland, OR}
  \end{Presentation}
  
  \begin{Presentation}{Sokol, E.R., {\bf N.I. Wisnoski}, C.M. Swan, R. Andrade, H.L. Bateman, A.G. Hope, J. Kominoski, N.K. Lany, L. Marazzi, S.J. Presley, A. Rassweiler, S. Record, M.R. Willig, and P.L.
Zarnetske}{2017}{The role of long-term ecological research programs for testing
metacommunity theory and understanding biodiversity patterns}{Ecological Society of
America Annual Meeting}{Portland, OR}
  \end{Presentation}
  
  \begin{Presentation}{Voelker, N.M., E.R. Sokol, {\bf N.I. Wisnoski}, C.M. Swan, T. Lamy, M.C.N. Castorani, L. Marazzi, A. Compagnoni, J.R. Blanchard, R. Andrade, and N.K. Lany}{2017}{Evaluating the link between metacommunity stability and environmental variability across trophic groups represented at LTER sites}{Ecological Society of America Annual Meeting}{Portland, OR}
  \end{Presentation}

  \begin{Presentation}{{\bf Wisnoski, N.I.} and J.T. Lennon}{2016}{Community assembly processes differ between surface water and sediment-associated communities in stream networks}{Ecological Society of America Annual Meeting}{Fort Lauderdale, FL}
  \end{Presentation}

  \begin{Presentation}{{\bf Wisnoski, N.I.} and J.T. Lennon}{2016}{Local and regional processes in stream microbial community assembly (poster)}{International Symposium on Microbial Ecology (ISME 16)}{Montreal, QC}
  \end{Presentation}

  \begin{Presentation}{{\bf Wisnoski, N.I.}, A.S. Ward, and J.T. Lennon}{2015}{Bacterial metacommunity structure across a stream network (poster)}{LTER All Scientists Meeting}{Estes Park, CO}
  \end{Presentation}

\end{rhangSection}

\bigskip

%-------------------------------------------------------------------------------
%	Teaching SECTION
%-------------------------------------------------------------------------------
\begin{rSection}{Teaching}

\begin{Course}
  {Co-Instructor}{Quantitative Biodiversity (\url{https://goo.gl/y4oK7c})}{Indiana University}{Spring 2017}
\end{Course}

\begin{Course}
  {Associate Instructor}{Foundations of Biology: Diversity, Evolution, and Ecology}{Indiana University}{Spring 2016, Fall 2016, Spring 2018, Spring 2019}
\end{Course}

\begin{Course}
  {Associate Instructor}{Biology Laboratory}{Indiana University}{Fall 2014, Fall 2017, Fall 2018}
\end{Course}

\begin{Course}
  {Grader}{Microbial Ecology}{University of Texas}{Spring 2014}
\end{Course}

\begin{Course}
  {Teaching Assistant}{Biostatistics}{University of Texas}{Spring 2013, Fall 2013}
\end{Course}

\end{rSection}

\bigskip

%-------------------------------------------------------------------------------
%	Mentorship Service SECTION
%-------------------------------------------------------------------------------
\begin{rSection}{Mentorship and Service}

\begin{rSubsection}{Undergraduate STEM Mentor}{Jan 2015 -- Present}{Indiana University}{}
\item Undergraduate Mentees: Luke Pryke, Mollie Carrison
\item Summer REU Program: Jaylen Beatty, Mary Wallace, SydneyEllen Gooding
\end{rSubsection}

\begin{rSubsection}{High School STEM Mentor}{Summer 2015 -- Present}{Jim Holland Summer Scholars Program}{}
\item High School Mentees: Dakayla Calhoun, Samuel Iwu, Ian Schowe
\end{rSubsection}

\begin{rSubsection}{High School Riverwatch Sampling Coordinator}{Summer 2017 -- Present}{Jim Holland Summer Enrichment Program}{}
\item Supervised macroinvertebrate sampling activity with high school students to analyze stream quality
\end{rSubsection}

\begin{rSubsection}{EcoLunch Co-Organizer}{August 2015 -- May 2016}{Indiana University}{}
\item Led student research and professional development seminar series 
\end{rSubsection}

\begin{rSubsection}{Metacommunity Reading Group Organizer}{Summer 2015}{Indiana University}{}
\item Organized a graduate reading group on metacommunity ecology
\end{rSubsection}

\end{rSection}

\bigskip

%-------------------------------------------------------------------------------
%	COMPUTATIONAL SKILLS SECTION
%-------------------------------------------------------------------------------
\begin{rSection}{Computational Skills}

\begin{tabular}{ @{} >{\bfseries}l @{\hspace{6ex}} l }
Scripting & {\tt python}, {\tt bash} \\
Analysis & {\tt R}, Mathematica \\
Productivity & \LaTeX, markdown, {\tt git}/GitHub, Microsoft Office \\
\end{tabular}

\end{rSection}

\bigskip

%-------------------------------------------------------------------------------
%	SOCIETY SECTION
%-------------------------------------------------------------------------------
\begin{rSection}{Professional Society Membership}

Ecological Society of America\\
Association for the Sciences of Limnology and Oceanography\\
Society for Freshwater Sciences\\

\end{rSection}

%-------------------------------------------------------------------------------
%-------------------------------------------------------------------------------
%	REFERENCES SECTION
%-------------------------------------------------------------------------------
%\begin{rSection}{References}
%
%Jay T. Lennon\\
%lennonj@indiana.edu\\
%Professor, Department of Biology\\
%Indiana University -- Bloomington, IN
%
%Mathew A. Leibold\\
%mleibold@ufl.edu\\
%Professor, Department of Biology\\
%University of Florida -- Gainsville
%
%Spencer R. Hall\\
%sprhall@indiana.edu\\
%Professor, Department of Biology\\
%Indiana University -- Bloomington
%
%\end{rSection}



\end{document}
