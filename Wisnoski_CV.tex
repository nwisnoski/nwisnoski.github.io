%%%%%%%%%%%%%%%%%%%%%%%%%%%%%%%%%%%%%%%%%
% Medium Length Professional CV
% LaTeX Template
% Version 2.0 (8/5/13)
%
% This template has been downloaded from:
% http://www.LaTeXTemplates.com
%
% Original author:
% Trey Hunner (http://www.treyhunner.com/)
%
% Important note:
% This template requires the resume.cls file to be in the same directory as the
% .tex file. The resume.cls file provides the resume style used for structuring the
% document.
%
%%%%%%%%%%%%%%%%%%%%%%%%%%%%%%%%%%%%%%%%%

%-------------------------------------------------------------------------------
%	PACKAGES AND OTHER DOCUMENT CONFIGURATIONS
%-------------------------------------------------------------------------------

\documentclass{resume} % Use the custom resume.cls style

\usepackage[left=.75in,top=1in,right=.75in,bottom=1in]{geometry} % Document margins
\usepackage{fancyhdr}
\usepackage{datetime}
\usepackage{eurosym}
\pagestyle{fancy}
\newdateformat{mydate}{\THEMONTH/\THEYEAR}
\fancyhf{} % sets both header and footer to nothing
\renewcommand{\headrulewidth}{0pt}
\rfoot{\mydate\today}
\cfoot{\thepage}



\name{\sc Nathan I. Wisnoski} % Your name
\address{Department of Biology, Indiana University \\ 1001 E. Third Street, Bloomington, IN 47405} % Your address
\address{(812) 856--7235 \\ wisnoski@indiana.edu} % Your phone number and email

\begin{document}

%-------------------------------------------------------------------------------
%	EDUCATION SECTION
%-------------------------------------------------------------------------------

\begin{rSection}{Education}

{\bf Indiana University, Bloomington} \hfill {2014 -- Present} \\
Ph.D in Biology -- Evolution, Ecology, and Behavior \\
Minor in Environmental Sciences, School of Public and Environmental Affairs

{\bf The University of Texas at Austin} \hfill {2009 -- 2013} \\
B.S. in Biology -- Ecology, Evolution, and Behavior \\
Minor in Business, McCombs School of Business \\

\end{rSection}

%-------------------------------------------------------------------------------
%	RESEARCH EXPERIENCE SECTION
%-------------------------------------------------------------------------------

\begin{rSection}{Research Experience}

% At Indiana University

\begin{rSubsection}{Graduate Researcher}{August 2014 -- Present}{Indiana University}{Bloomington, IN}
\item Lennon Lab: Microbial Ecology and Evolution
\end{rSubsection}

\begin{rSubsection}{Field/Lab Technician}{Spring 2014}{University of Texas}{Austin, TX}
\item Hawkes Lab: Microbial Ecology and Biogeochemistry
\end{rSubsection}

\begin{rSubsection}{Undergraduate Researcher}{June 2012 -- February 2014}{University of Texas}{Austin, TX}
\item Leibold Lab: Community and Evolutionary Ecology
\end{rSubsection}

\end{rSection}

%-------------------------------------------------------------------------------
%   POSTERS / PRESENTATIONS
%-------------------------------------------------------------------------------

\begin{rSection}{Posters and Presentations}

  \begin{Publication}{{\bf Wisnoski NI}, Lennon JT}{2016}{Community assembly processes differ between surface water and sediment-associated communities in stream networks}{Annual Meeting of the Ecological Society of America, Fort Lauderdale, FL}
  \end{Publication}
  
  \begin{Publication}{{\bf Wisnoski NI}, Lennon JT}{2016}{Local and regional processes in stream microbial community assembly}{International Symposium on Microbial Ecology (ISME 16), Montreal, QC}
  \end{Publication}

  \begin{Publication}{\bf Wisnoski NI}{2015}{Stream Microbial Metacommunities}{EcoLunch, Bloomington, IN}
  \end{Publication}

  \begin{Publication}{{\bf Wisnoski NI}, Ward AS, Lennon JT}{2015}{Bacterial metacommunity structure across a stream network}{LTER All Scientists Meeting, Estes Park, CO;  IU Microbiology Retreat, Nashville, IN}
  \end{Publication}

  \begin{Publication}{\bf Wisnoski NI}{2015}{Using the metacommunity concept to synthesize biodiversity patterns across LTER sites}{Working group presenter and participant, LTER All Scientists Meeting, Estes Park, CO}

  \end{Publication}


\end{rSection}

%-------------------------------------------------------------------------------
%   FUNDING/AWARDS/Fellowships
%-------------------------------------------------------------------------------

\begin{rSection}{Funding}

\begin{Grant}{George W. Brackenridge Fellowship, IU Biology}{\$2000}{2016}
\end{Grant}

\begin{Grant}{Travel Grant, International Society for Microbial Ecology}{\euro{}300}{2016}
\end{Grant}

\begin{Grant}{Sustainability Research Development Grant, IU Sustainability}{\$5400}{2015}
\end{Grant}

\begin{Grant}{Departmental Research Recruitment Fellowship, IU Biology}{ }{2014}
\end{Grant}

\end{rSection}

%-------------------------------------------------------------------------------
%	Teaching SECTION
%-------------------------------------------------------------------------------
\begin{rSection}{Teaching}

\begin{rSubsection}{Associate Instructor}{August 2014 -- Present}{Indiana University}{Bloomington, IN}
\item Courses Taught: Introductory Biology Lecture (L111) and Lab (L113)
\end{rSubsection}

\begin{rSubsection}{Grader}{January 2014 -- May 2014}{University of Texas}{Austin, TX}
\item Course: Microbial Ecology
\end{rSubsection}

\begin{rSubsection}{Undergraduate Teaching Assistant}{January 2013 -- December 2013}{University of Texas}{Austin, TX}
\item Courses Taught: Biostatistics
\end{rSubsection}

\end{rSection}

%-------------------------------------------------------------------------------
%	Mentorship Service SECTION
%-------------------------------------------------------------------------------
\begin{rSection}{Mentorship and Service}

\begin{rSubsection}{EcoLunch Co-Organizer}{August 2015 -- Present}{Indiana University}{Bloomington, IN}
\item Redesigned and led student-run research and professional development seminar/discussion series for Ecology-focused labs at IU
\end{rSubsection}

\begin{rSubsection}{Undergraduate STEM Mentor}{Jan 2015 -- Present}{Indiana University}{Bloomington, IN}
\item Undergraduate Mentee: Luke Pryke. Luke is investigating the implications of microbial dormancy in predator-prey interactions using the model organisms \emph{Bacillus subtilis} and \emph{Tetrahymena thermophila}.
\end{rSubsection}

\begin{rSubsection}{Metacommunity Discussion Group Leader}{Summer 2015}{Indiana University}{Bloomington, IN}
\item Organized a graduate reading group to discuss the theory and applications of the metacommunity concept in ecology
\end{rSubsection}

\begin{rSubsection}{High School STEM Mentor}{Summer 2015}{Indiana University}{Bloomington, IN}
\item High School Mentee: Dakayla Calhoun. Through IU's Jim Holland Summer Scholars Program, Dakayla identified changes in resource utilization across a variety of bacterial strains after years of long-term starvation. GitHub repository for the project located at {\tt https://github.com/LennonLab/}
\end{rSubsection}

\begin{rSubsection}{EcoLunch Committee}{August 2014 -- May 2015}{Indiana University}{Bloomington, IN}
\item Co-organized the student-run research and professional development seminar series
\end{rSubsection}

\end{rSection}

%-------------------------------------------------------------------------------
%	COMPUTATIONAL SKILLS SECTION
%-------------------------------------------------------------------------------

\begin{rSection}{Computational Skills}

\begin{tabular}{ @{} >{\bfseries}l @{\hspace{6ex}} l }
Scripting & {\tt python}, {\tt bash} \\
Analysis & R, Mathematica, MATLAB \\
Productivity & \LaTeX, markdown, {\tt git}/GitHub, Microsoft Office \\
\end{tabular}

\end{rSection}

%-------------------------------------------------------------------------------
%	SOCIETY SECTION
%-------------------------------------------------------------------------------

\begin{rSection}{Professional Society Membership}

Ecological Society of America\\
International Society for Microbial Ecology

\end{rSection}

%-------------------------------------------------------------------------------

\end{document}
